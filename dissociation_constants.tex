\documentclass[a4paper, 10pt]{article}

\usepackage[top=1cm, bottom=2cm, left=1cm, right=1cm, includefoot]{geometry}

\label{conf_SI}
\usepackage[separate-uncertainty = true,multi-part-units=single]{siunitx}
\sisetup{output-decimal-marker = {.}, round-mode=off, add-decimal-zero=false}

\label{conf_spacing_par}
\usepackage{xspace}
\usepackage{setspace}
\singlespacing
\parindent 0ex

\label{conf_chemistry_packages}
\usepackage[version=4]{mhchem}
\usepackage{chemmacros}
\chemsetup[acid-base]{p-style=slanted}
\NewChemEqConstant{\Kc}{K-conc}{\mathrm{c}}
\NewChemEqConstant{\Kp}{K-press}{\mathrm{p}}
\usechemmodule{thermodynamics}

\label{conf_table}
\usepackage{booktabs}
\usepackage{supertabular}

\label{conf_head_footnote}
\usepackage[absolute]{textpos}
\usepackage{fancyhdr}
\pagestyle{fancy}
\fancyhead{}
\fancyhead[L]{}
\fancyhead[C]{Dissociation constants \\ Francisco Bustamante}
\fancyhead[R]{}
\setlength{\headheight}{55pt}
\fancyfoot{}
\renewcommand{\footrulewidth}{1pt}
\fancyfoot[R]{\footnotesize Page \thepage \,of \pageref*{LastPage}}
\renewcommand{\thefootnote}{\fnsymbol{footnote}}

\label{conf_refs}
\usepackage{lastpage}
\usepackage{hyperref}

\label{conf_id_doc}
\author{Francisco Bustamante}
\title{\textit{Dissociation constants}}
\date{2019}

\begin{document}

\small

\twocolumn

The data in these tables are presented as values of \pKa, defined as the negative
logarithm of the acid dissociation constant \Ka\, for the reaction \ce{BH <=> B- + H+}.

Thus $\pKa = -\log \Ka$, and the hydrogen ion concentration \ce{[H+]} can be
calculated from
%
\begin{align*}
    \Ka = \frac{\ce{[H^+][B^-]}}{\ce{[BH]}}
\end{align*}
%
In the case of bases, the entry in the table is for the conjugate acid; e.g.,
ammonium ion for ammonia. The \ce{OH-} concentration in the system
\ce{NH3 + H2O <=> NH4+ + OH-} can be calculated from the equation
%
\begin{align*}
    \Kb = \frac{K_{water}}{\Ka} = \frac{\ce{[OH^-][NH4^+]}}{\ce{[NH3]}}
\end{align*}
%
where $K_{water} = \num{1.01d-14}$ at \SI{25}{\celsius}. Note that
$\pKa + \pKb = pK_{water}$. All values refer to dilute aqueous solutions at zero
ionic strength at the temperature indicated. The tables are arranged
alphabetically by compound name.

Reference: Rumble, J. \textit{CRC Handbook of Chemistry and Physics}, 98th
Edition, CRC Press LLC, 2017.

\begin{center}
    \begin{large}
        \textbf{Inorganic acids and bases}
    \end{large}
\end{center}

\tablehead{%
    \toprule
    Name                            & Formula      & Step & T/\si{\celsius} & pKa \\
    \midrule
}

\begin{small}
    \begin{supertabular}{llccl}
        Aluminum ion \ce{[Al^{+3}]} & \ce{Al^{+3}} &      & 25              & \num{5.0} \\
               Ammonia              & \ce{NH3}     &      & 25              & \num{9.25} \\
          Arsenic acid              & \ce{H3AsO4}  & 1    & 25              & \num{2.26} \\
                                    & \ce{}        & 2    & 25              & \num{6.76} \\
                                    & \ce{}        & 3    & 25              & \num{11.29} \\
        Arsenious acid              & \ce{H3AsO3}  &      & 25              & \num{9.29} \\
     Barium ion \ce{[Ba^{+2}]}      & \ce{Ba^{+2}} &      & 25              & \num{13.4} \\
            Boric acid              & \ce{H3BO3}   & 1    & 20              & \num{9.27} \\
                                    & \ce{}        & 2    & 20              & \num{14} \\
    Calcium ion \ce{[Ca^{+2}]}      & \ce{Ca^{+2}} &      & 25              & \num{12.6} \\
         Carbonic acid              & \ce{H2CO3}   & 1    & 25              & \num{6.35} \\
                                    & \ce{}        & 2    & 25              & \num{10.33} \\
         Chlorous acid              & \ce{HClO2}   &      & 25              & \num{1.94} \\
          Chromic acid              & \ce{H2CrO4}  & 1    & 25              & \num{0.74} \\
                                    & \ce{}        & 2    & 25              & \num{6.49} \\
           Cyanic acid              & \ce{HOCN}    &      & 25              & \num{3.46} \\
     Diphosphoric acid              & \ce{H4P2O7}  & 1    & 25              & \num{0.91} \\
                                    & \ce{}        & 2    & 25              & \num{2.10} \\
                                    & \ce{}        & 3    & 25              & \num{6.70} \\
                                    & \ce{}        & 4    & 25              & \num{9.32} \\
         Germanic acid              & \ce{H2GeO3}  & 1    & 25              & \num{9.01} \\
                                    & \ce{}        & 2    & 25              & \num{12.3} \\
             Hydrazine              & \ce{N2H4}    &      & 25              & \num{8.1} \\
        Hydrazoic acid              & \ce{HN3}     &      & 25              & \num{4.6} \\
      Hydrogen cyanide              & \ce{HCN}     &      & 25              & \num{9.21} \\
     Hydrogen fluoride              & \ce{HF}      &      & 25              & \num{3.20} \\
     Hydrogen peroxide              & \ce{H2O2}    &      & 25              & \num{11.62} \\
     Hydrogen selenide              & \ce{H2Se}    & 1    & 25              & \num{3.89} \\
                                    & \ce{}        & 2    & 25              & \num{11.0} \\
      Hydrogen sulfide              & \ce{H2S}     & 1    & 25              & \num{7.05} \\
                                    & \ce{}        & 2    & 25              & \num{19} \\
    Hydrogen telluride              & \ce{H2Te}    & 1    & 18              & \num{2.6} \\
                                    & \ce{}        & 2    & 25              & \num{11} \\
         Hydroxylamine              & \ce{H2NOH}   &      & 25              & \num{5.94} \\
      Hypobromous acid              & \ce{HOBr}    &      & 25              & \num{8.55} \\
     Hypochlorous acid              & \ce{HOCl}    &      & 25              & \num{7.40} \\
       Hypoiodous acid              & \ce{HIO}     &      & 25              & \num{10.5} \\
            Iodic acid              & \ce{HIO3}    &      & 25              & \num{0.78} \\
     Lithium ion \ce{[Li^{+}]}      & \ce{Li^{+}}  &      & 25              & \num{13.8} \\
  Magnesium ion \ce{[Mg^{+2}]}      & \ce{Mg^{+2}} &      & 25              & \num{11.4} \\
          Nitrous acid              & \ce{HNO2}    &      & 25              & \num{3.25} \\
     Orthosilicic acid              & \ce{H4SiO4}  & 1    & 30              & \num{9.9} \\
                                    & \ce{}        & 2    & 30              & \num{11.8} \\
                                    & \ce{}        & 3    & 30              & \num{12} \\
                                    & \ce{}        & 4    & 30              & \num{12} \\
       Perchloric acid              & \ce{HClO4}   &      & 20              & \num{-1.6} \\
         Periodic acid              & \ce{HIO4}    &      & 25              & \num{1.64} \\
       Phosphonic acid              & \ce{H3PO3}   & 1    & 20              & \num{1.3} \\
                                    & \ce{}        & 2    & 20              & \num{6.70} \\
       Phosphoric acid              & \ce{H3PO4}   & 1    & 25              & \num{2.16} \\
                                    & \ce{}        & 2    & 25              & \num{7.21} \\
                                    & \ce{}        & 3    & 25              & \num{12.32} \\
          Selenic acid              & \ce{H2SeO4}  & 2    & 25              & \num{1.7} \\
         Selenous acid              & \ce{H2SeO3}  & 1    & 25              & \num{2.62} \\
                                    & \ce{}        & 2    & 25              & \num{8.32} \\
      Sodium ion \ce{[Na^{+}]}      & \ce{Na^{+}}  &      & 25              & \num{14.8} \\
  Strontium ion \ce{[Sr^{+2}]}      & \ce{Sr^{+2}} &      & 25              & \num{13.2} \\
         Sulfamic acid              & \ce{H2NSO3H} &      & 25              & \num{1.05} \\
         Sulfuric acid              & \ce{H2SO4}   & 2    & 25              & \num{1.99} \\
        Sulfurous acid              & \ce{H2SO3}   & 1    & 25              & \num{1.85} \\
                                    & \ce{}        & 2    & 25              & \num{7.2} \\
     Telluric(VI) acid              & \ce{H6TeO6}  & 1    & 18              & \num{7.68} \\
                                    & \ce{}        & 2    & 18              & \num{11.0} \\
        Tellurous acid              & \ce{H2TeO3}  & 1    & 25              & \num{6.27} \\
                                    & \ce{}        & 2    & 25              & \num{8.43} \\
 Tetrafluoroboric acid              & \ce{HBF4}    &      & 25              & \num{0.5} \\
       Thiocyanic acid              & \ce{HCNS}    &      & 25              & \num{-1.8} \\
                 Water              & \ce{H2O}     &      & 25              & \num{13.995} \\
    \bottomrule
    \end{supertabular}
\end{small}

\begin{center}
    \begin{large}
        \textbf{Organic acids and bases}
    \end{large}
\end{center}

\tablehead{%
    \toprule
    Name                     & Formula            & Step & T/\si{\celsius} & pKa \\
    \midrule
}

\begin{footnotesize}
    \begin{supertabular}{llc@{\hspace{2.0mm}}c@{\hspace{2.0mm}}l}
        Acetic acid          & \ce{CH3COOH}       &      & 25              & \num{4,756} \\
        L-ascorbic acid      & \ce{C6H6O6}        & 1    & 25              & \num{4,04} \\
                             &                    & 2    & 16              & \num{11,7} \\
        Aniline              & \ce{C6H5NH2}       &      & 25              & \num{4,87} \\
        Benzenesulfonic acid & \ce{C6H5SO3H}      &      & 25              & \num{0,70} \\
        Benzoic acid         & \ce{C6H5COOH}      &      & 25              & \num{4,204} \\
        Citric acid          & \ce{H3C6H5O7}      & 1    & 25              & \num{3,13} \\
                             &                    & 2    & 25              & \num{4,76} \\
                             &                    & 3    & 25              & \num{6,40} \\
        Chloroacetic acid    & \ce{CH2ClCOOH}     &      & 25              & \num{2,85} \\
        Dimethylamine        & \ce{(CH3)2NH}      &      & 25              & \num{10,73} \\
        Ethylamine           & \ce{C2H5NH2}       &      & 25              & \num{10,65} \\
        Ethylenediamine      & \ce{H2NCH2CH2NH2}  & 1    & 25              & \num{9,92} \\
                             &                    & 2    & 25              & \num{6,86} \\
        Formic acid          & \ce{HCO2H}         &      & 25              & \num{3,75} \\
        D-Lactic acid        & \ce{CH3CH(OH)COOH} &      & 25              & \num{3,86} \\
        Methylamine          & \ce{CH3NH2}        &      & 25              & \num{10,66} \\
        Morphine             & \ce{C17H19NO3}     & 1    & 25              & \num{8,21} \\
                             &                    & 2    & 20              & \num{9,85} \\
        L-Nicotine           & \ce{C10H14N2}      & 1    & 25              & \num{8,02} \\
                             &                    & 2    & 25              & \num{3,12} \\
        Oxalic acid          & \ce{HOOCCOOH}      & 1    & 25              & \num{1,25} \\
                             &                    & 2    & 25              & \num{3,81} \\
        Phenol               & \ce{C6H5OH}        &      & 25              & \num{9,99} \\
        Pyridine             & \ce{C5H5N}         &      & 25              & \num{5,23} \\
        Trichloroacetic acid & \ce{CCl3COOH}      &      & 25              & \num{0,66} \\
        Trimethylamine       & \ce{(CH3)3N}       &      & 25              & \num{9,80} \\
        Urea                 & \ce{CO(NH2)2}      &      & 25              & \num{0,10} \\
    \bottomrule
    \end{supertabular}
\end{footnotesize}

\end{document}