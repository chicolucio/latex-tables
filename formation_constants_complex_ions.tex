\documentclass[a4paper, 10pt]{article}

%margens
\usepackage[top=1cm, bottom=2cm, left=1cm, right=1cm, includefoot]{geometry}

\label{conf_geral}
\usepackage[utf8x]{inputenc} %caracteres acentuados
\usepackage{color}
\usepackage[usenames,dvipsnames,svgnames,table]{xcolor}
\usepackage[brazil]{babel}
\usepackage{amsfonts, amssymb, amsmath, amsthm, commath, physics} % ver explicação em https://tex,stackexchange,com/questions/32100/what-does-each-ams-package-do
\usepackage{mathtools}
\usepackage{proof} %?
%\usepackage{eulervm, bookman}
\usepackage{marginnote} % nota de margem
\usepackage{comment} % comentários longos no código
\usepackage[font=small,labelfont=bf]{caption} %fonte das legendas
\usepackage{lipsum} %gerador de lorota em Latim
\usepackage{lscape}
\usepackage{pdflscape}

\label{conf_fonts}
\usepackage[fontsize=10pt]{scrextend}
%\usepackage{moresize}
\usepackage[normalem]{ulem}

\label{conf_graph}
\usepackage{graphicx} %inserir figuras
\usepackage{float}
\usepackage{tikz}
\usepackage{wrapfig} %figuras rodeadas por texto

\label{conf_SI}
\usepackage[separate-uncertainty = true,multi-part-units=single]{siunitx}
\DeclareSIUnit{\atmosphere}{atm}
\DeclareSIUnit{\calorie}{cal}
\sisetup{output-decimal-marker = {.}, round-mode=off, add-decimal-zero=false} %números na forma 1,0 ao invés de 1,0

\label{conf_spacing_par}
\usepackage{xspace} %para correção de espaços
\usepackage{setspace} %espaçamento entre linhas e opções mais comuns
\singlespacing
%\onehalfspacing
%\doublespacing
%\setstretch{1,1}
\parindent 0ex %retirada de indentação de todos os parágrafos
%\setlength\parskip{\baselineskip}
\usepackage{parskip} %espaçamento entre parágrafos

\label{conf_chemistry_packages}
\usepackage[version=4]{mhchem} %pacote química
\usepackage{chemfig}
\renewcommand*\printatom[1]{\ensuremath{\mathsf{#1}}}
\usepackage{elements}
\usepackage{chemmacros}
\chemsetup[acid-base]{p-style=slanted}
\NewChemEqConstant{\Kc}{K-conc}{\mathrm{c}}
\NewChemEqConstant{\Kp}{K-press}{\mathrm{p}}
\usechemmodule{thermodynamics}

\label{conf_table}
\usepackage{array} %tabelas
\usepackage{booktabs}
\usepackage{makecell} %editar células em tabelas
\renewcommand\theadalign{cb}
\renewcommand\theadfont{\bfseries}
\renewcommand\theadgape{\Gape[4pt]}
\renewcommand\cellgape{\Gape[1pt]}
\usepackage{multirow}
\usepackage{colortbl}
\usepackage{longtable}
\usepackage{multicol}
\usepackage{supertabular}

\label{conf_head_footnote}
\usepackage[absolute]{textpos}
\usepackage{fancyhdr} %cabeçalhos e rodapé
\pagestyle{fancy}
\fancyhead{}
\fancyhead[L]{}
\fancyhead[C]{Formation constants for some complex ions em aqueous solutions at \SI{25}{\celsius} \\ Francisco Bustamante}
\fancyhead[R]{}
\setlength{\headheight}{55pt}
\fancyfoot{}
\renewcommand{\footrulewidth}{1pt}
\fancyfoot[R]{\footnotesize Page \thepage \,of \pageref*{LastPage}}
\renewcommand{\thefootnote}{\fnsymbol{footnote}}

\label{conf_xlop}
\usepackage{xlop}
\opset{decimalsepsymbol={,}}

\label{conf_refs}
\usepackage[super,sort&compress]{natbib} %numeração refs sobrescrito
\usepackage{notoccite}
\usepackage{lastpage} %possibilita se referir à última página
\setcounter{secnumdepth}{3}
\usepackage{hyperref} % referências cruzadas

\label{conf_id_doc}
%identificação do documento
\author{Prof. Francisco Bustamante}
\title{\textit{Formation constants for some complex ions em aqueous solutions at \SI{25}{\celsius}Solubility product constants}}
\date{2019}

\begin{document}

\small

\twocolumn

Reference: Dean, J. A. \textit{Lange's Handbook of Chemistry}, 15th Edition, New York: McGraw-Hill Publishers, 1999.

\tablehead{%
    \toprule
    Formation Equilibrium &             $K$  \\
    \midrule
}

\begin{footnotesize}
    \begin{supertabular}{ll}
    \ce{Ag+ + 2Br- <=> [AgBr2]-} &   \num{2.14d7} \\
    \ce{Ag+ + 2Cl- <=> [AgCl2]-} &   \num{1.09d5} \\
    \ce{Ag+ + 2CN- <=> [Ag(CN)2]-} &  \num{1.26d21} \\
    \ce{Ag+ + 2S2O3^{2-} <=> [Ag(S2O3)2]^{3-}} &  \num{2.88d13} \\
    \ce{Ag+ + 2NH3 <=> [Ag(NH3)2]+} &   \num{1.12d7} \\
    \ce{Al^{3+} + 6F- <=> [AlF6]^{3-}} &  \num{6.92d19} \\
    \ce{Al^{3+} + 4OH- <=> [Al(OH)4]-} &  \num{1.07d33} \\
    \ce{Au+ + 2CN- <=> [Au(CN)2]-} &  \num{2.00d38} \\
    \ce{Cd^{2+} + 4CN- <=> [Cd(CN)4]^{2-}} &  \num{6.02d18} \\
    \ce{Cd^{2+} + 4I- <=> [CdI4]^{2-}} &  \num{2.57d5} \\
    \ce{Cd^{2+} + 4NH3 <=> [Cd(NH3)4]^{2+}} &   \num{1.32d7} \\
    \ce{Co^{2+} + edta <=> [Co(edta)]^{2+}} & \num{2.04d16} \\
    \ce{Co^{2+} + 3en <=> [Co(en)3]^{2+}} & \num{8.71d13} \\
    \ce{Co^{2+} + 6NH3 <=> [Co(NH3)6]^{2+}} &   \num{1.29d5} \\
    \ce{Co^{3+} + edta <=> [Co(edta)]^{3+}} & \num{1d36} \\
    \ce{Co^{3+} + 3en <=> [Co(en)3]^{3+}} & \num{4.90d48} \\
    \ce{Co^{3+} + 6NH3 <=> [Co(NH3)6]^{3+}} &   \num{1.58d35} \\
    \ce{Cr^{2+} + edta <=> [Cr(edta)]^{2+}} & \num{3.98d13} \\
    \ce{Cr^{3+} + edta <=> [Cr(edta)]^{3+}} & \num{1d23} \\
    \ce{Cu+ + 2CN- <=> [Cu(CN)2]-} &  \num{1.00d24} \\
    \ce{Cu+ + 2Cl- <=> [CuCl2]-} &   \num{3.16d5} \\
    \ce{Cr^{3+} + 4OH- <=> [Cr(OH)4]-} &  \num{7.94d29} \\
    \ce{Cu^{2+} + edta <=> [Cu(edta)]^{2+}} &  \num{5.01d18} \\
    \ce{Cu^{2+} + 4NH3 <=> [Cu(NH3)4]^{2+}} &  \num{2.09d13} \\
    \ce{Fe^{2+} + 6CN- <=> [Fe(CN)6]^{4-}} &  \num{1d35} \\
    \ce{Fe^{2+} + edta <=> [Fe(edta)]^{2+}} & \num{2.14d14} \\
    \ce{Fe^{2+} + 3en <=> [Fe(en)3]^{2+}} & \num{5.01d9} \\
    \ce{Fe^{3+} + 6CN- <=> [Fe(CN)6]^{3-}} &  \num{1d42} \\
    \ce{Fe^{3+} + edta <=> [Fe(edta)]^{3+}} & \num{1.70d24} \\
    \ce{Hg^{2+} + 4Cl- <=> [HgCl4]^{2-}} &  \num{1.17d15} \\
    \ce{Hg^{2+} + 4CN- <=> [Hg(CN)4]^{2-}} &  \num{2.51d41} \\
    \ce{Hg^{2+} + edta <=> [Hg(edta)]^{2+}} & \num{6.31d21} \\
    \ce{Hg^{2+} + 4NH3 <=> [Hg(NH3)4]^{2+}} &  \num{1.90d19} \\
    \ce{Mn^{2+} + edta <=> [Mn(edta)]^{2+}} & \num{6.31d13} \\
    \ce{Ni^{2+} + 4CN- <=> [Ni(CN)4]^{2-}} &  \num{2.00d31} \\
    \ce{Ni^{2+} + 6NH3 <=> [Ni(NH3)6]^{2+}} &   \num{5.49d8} \\
    \ce{Pb^{2+} + 4I- <=> [PbI4]^{2-}} &  \num{2.95d4} \\
    \ce{Pd^{2+} + 4Cl- <=> [PdCl4]^{2-}} &  \num{5.01d15} \\
    \ce{Pt^{2+} + 4Cl- <=> [PtCl4]^{2-}} &  \num{1.00d16} \\
    \ce{Pt^{2+} + 6NH3 <=> [Pt(NH3)6]^{2+}} &   \num{2.00d35} \\
    \ce{Zn^{2+} + edta <=> [Zn(edta)]^{2+}} & \num{2.51d16} \\
    \ce{Zn^{2+} + 3en <=> [Zn(en)3]^{2+}} & \num{1.29d14} \\
    \ce{Zn^{2+} + 4OH- <=> [Zn(OH)4]^{2-}} &  \num{4.57d17} \\
    \ce{Zn^{2+} + 4NH3 <=> [Zn(NH3)4]^{2+}} &   \num{2.88d9} \\
    \bottomrule
    \end{supertabular}
\end{footnotesize}

en $=$ ethylenediamine

edta $=$ ethylenediamine-\textit{N,N,N',N'}-tetraacetic acid

\end{document}